% ---- ETD Document Class and Useful Packages ---- %
% v1.2.0 released July 7, 2022.
% https://github.com/k4rtik/uchicago-dissertation
\documentclass{ucetd}

\usepackage[T1]{fontenc}
\usepackage{subcaption,graphicx}
\usepackage{natbib}
\usepackage{mathtools}  % loads amsmath
\usepackage{amssymb}    % loads amsfonts
\usepackage{amsthm}

%% Use these commands to set biographic information for the title page:
\newcommand{\thesistitle}{Thesis Title}
\newcommand{\thesisauthor}{Thesis Author}
\department{Thesis Department}
\division{Thesis Division}
\degree{Type of Degree}
\date{Graduation Date}

\title{\thesistitle}
\author{\thesisauthor}

%% Use these commands to set a dedication and epigraph text
\dedication{Dedication Text}
\epigraph{Epigraph Text}

\usepackage{doi}
\usepackage{xurl}
\hypersetup{bookmarksnumbered,
            unicode,
            linktoc=all,
            pdftitle={\thesistitle},
            pdfauthor={\thesisauthor},
            pdfsubject={},                                % Add subject/description
            % pdfkeywords={keyword1, keyword2, keyword3}, % Uncomment and revise keywords
            pdfborder={0 0 0}}
% See https://github.com/k4rtik/uchicago-dissertation/issues/1
\makeatletter
\let\ORG@hyper@linkstart\hyper@linkstart
\protected\def\hyper@linkstart#1#2{%
  \lowercase{\ORG@hyper@linkstart{#1}{#2}}}
\makeatother

\begin{document}
%% Basic setup commands
% If you don't want a title page comment out the next line and uncomment the line after it:
\maketitle
%\omittitle

% These lines can be commented out to disable the copyright/dedication/epigraph pages
\makecopyright
\makededication
\makeepigraph


%% Make the various tables of contents
\tableofcontents
\listoffigures
\listoftables

\acknowledgments
% Enter Acknowledgements here

\abstract
% Enter Abstract here

\mainmatter
% Main body of text follows

\chapter{Introduction}
% Introductory stuff

\chapter{A Chapter}
\section{Introduction}
% Intro to chapter one

% Format a LaTeX bibliography
\makebibliography
\nocite{*}

% Figures and tables, if you decide to leave them to the end
%\input{figure}
%\input{table}

\end{document}
